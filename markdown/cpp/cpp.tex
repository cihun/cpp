\documentclass[UTF8]{ctexrep}

\bibliographystyle{plain}

\title{C++}
\usepackage[hidelinks]{hyperref}
\usepackage{listings}
\usepackage{amsmath}
\usepackage{amssymb}
\usepackage{listings}
\author{}
\date{}

\lstset{ language={[ANSI]C++}}


\newtheorem{thm}{定理}[section]
\newtheorem{definition}{定义}[section]
\newtheorem{prop}{命题}[section]

\begin{document}

\maketitle
\tableofcontents
\newpage

\chapter{复合类型}
\section{string类操作}
\begin{enumerate}
    \item s1+s2 string类的合并,类似于数组字符串的strcpy()函数。
    \item s1+=s2 string加,类似于数组字符串的strcat()函数。
\end{enumerate}

使用string读取字符串:
\begin{enumerate}
    \item 读取一个单词时
    \begin{lstlisting}
        cin >> str1;
    \end{lstlisting}
    \item 读取一行时
    \begin{lstlisting}
        getline(cin,str1);
    \end{lstlisting}
\end{enumerate}

原始字符串,字符串中没有转义字符。结构为
R"(---)" or R"+*(---)+*",其中+*可以是其他任意数量的字符(斜杠除外)


\section{共用体union}
用法:
\begin{lstlisting}
    union one4all{
        int int_val;
        long long_val;
        double double_val;
    };
\end{lstlisting}

可以将共用体匿名放置在结构体中:
\begin{lstlisting}
    struct widget{
        char brand[20];
        int type; //用来标记共用体成员使用情况
        union{
            long id_num;
            char in_char[20];
        };
    };

    widget prize;
    //调用:
    prize.id_num;
    prize.id_char;
\end{lstlisting}

\section{enum枚举}
例:
\begin{lstlisting}
    enum spectrum{red,orange,yellow,green,bule,violet,indigo,utraviolet};
\end{lstlisting}
\begin{enumerate}
    \item 让spectrum成为新类型的名称;spectrum被称为枚举(enumeration)。
    \item 将red,orange,yellow等作为符号常量,他们对应整数值0~7。这些常量叫做枚举量
(enumerator)。
\end{enumerate}

可以使用刚才的枚举名声明变量,变量只能取枚举量。枚举没有定义算数运算。枚举量
是整形,可以被提升为int类型,但是int类型不能自动转换为枚举类型。但是可以通过强制转换
使int变成枚举类型。








%\begin{lstlisting}[caption={example code}]
%    #include<iosream>
%    using namespace std;
%    int main() {
%        cout << "Hello World!";
%        cin.get();
%        cin.get();
%        return 0;
%    }
%\end{lstlisting}




\end{document}